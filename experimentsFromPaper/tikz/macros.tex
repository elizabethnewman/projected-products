
% packages
\usepackage{amsmath, amsfonts} % basic math packages
\usepackage{tikz, pgfplots, pgflibraryplotmarks, pgfkeys, ifthen} % for making illustrations
\usetikzlibrary{shapes.arrows, arrows, decorations.markings}
\usetikzlibrary{calc, positioning}
\usepackage{graphicx} % for importing images
\usepackage{xcolor} % more color options
\usepackage{multicol} % for making two-column lists
\usepackage{hyperref} % for hyperlinking
%\hypersetup{colorlinks=true, urlcolor=cyan,}
\usepackage{framed}
\usepackage{cleveref}
\usepackage{xfrac}
\usepackage{mathabx}
% \usepackage{graphicx}
% \usepackage{tikz, pgfplots, pgflibraryplotmarks, pgfkeys, ifthen}
% \usepackage{pgf-pie} 
% \usetikzlibrary{calc}
% \usepackage{amssymb, amsfonts, amsmath, latexsym, dsfont}
% \usepackage{comment}
% \usepackage{array}
\pgfplotsset{compat = 1.3}
% ============================================================ %
% Macros
% ============================================================ %
%\newtheorem{example}[theorem]{Example}
%\theoremstyle{remark}
%\newtheorem*{remark}{Remark}



\definecolor{EmoryBlue}{RGB}{1, 33, 105} 
\definecolor{EmoryDarkBlue}{RGB}{12, 35, 64} 
\definecolor{EmoryMediumBlue}{RGB}{0, 51, 160} 
\definecolor{EmoryLightBlue}{RGB}{0, 125, 186} 
\definecolor{EmoryYellow}{RGB}{242, 169, 0} 
\definecolor{EmoryGold}{RGB}{181, 133, 0} 
\definecolor{EmoryMetallicGold}{RGB}{132, 117, 78} 


\definecolor{mycolor0}{rgb}{1, 1, 1}%
\definecolor{mycolor1}{rgb}{0.00000,0.44700,0.74100}%
\definecolor{mycolor2}{rgb}{0.85000,0.32500,0.09800}%
\definecolor{mycolor3}{rgb}{0.92900,0.69400,0.12500}%
\definecolor{mycolor4}{rgb}{0.49400,0.18400,0.55600}%
\definecolor{mycolor5}{rgb}{0.46600,0.67400,0.18800}%
\definecolor{mycolor6}{rgb}{0.30100,0.74500,0.93300}%

\definecolor{jet1}{rgb}{0.0000,0.0000,0.6667}
\definecolor{jet2}{rgb}{0.0000,0.0000,1.0000}
\definecolor{jet3}{rgb}{0.0000,0.3333,1.0000}
\definecolor{jet4}{rgb}{0.0000,0.6667,1.0000}
\definecolor{jet5}{rgb}{0.0000,1.0000,1.0000}
\definecolor{jet6}{rgb}{0.3333,1.0000,0.6667}
\definecolor{jet7}{rgb}{0.6667,1.0000,0.3333}
\definecolor{jet8}{rgb}{1.0000,1.0000,0.0000}
\definecolor{jet9}{rgb}{1.0000,0.6667,0.0000}
\definecolor{jet10}{rgb}{1.0000,0.3333,0.0000}
\definecolor{jet11}{rgb}{1.0000,0.0000,0.0000}



\definecolor{mygreen}{RGB}{111, 113, 88}

\usepackage[most]{tcolorbox}
\usepackage{cleveref}
% \tcbuselibrary{theorems}


\makeatletter
%\crefname{tcb@cnt@mytheo}{theorem}{theorem}
%\Crefname{tcb@cnt@mytheo}{Theorem}{Theorem}
\crefformat{tcb@cnt@mytheo}{theorem~#2#1#3}
\Crefformat{tcb@cnt@mytheo}{Theorem~#2#1#3}
\crefformat{tcb@cnt@mylemma}{lemma~#2#1#3}
\Crefformat{tcb@cnt@mylemma}{Lemma~#2#1#3}
\crefformat{tcb@cnt@mycorollary}{corollary~#2#1#3}
\Crefformat{tcb@cnt@mycorollary}{Corollary~#2#1#3}
\crefformat{tcb@cnt@myproto}{prototype problem~#2#1#3}
\Crefformat{tcb@cnt@myproto}{Prototype Problem~#2#1#3}
\crefformat{tcb@cnt@myalg}{algorithm~#2#1#3}
\Crefformat{tcb@cnt@myalg}{Algorithm~#2#1#3}
\crefformat{tcb@cnt@myexam}{example~#2#1#3}
\Crefformat{tcb@cnt@myexam}{Example~#2#1#3}
%\crefname{tcb@cnt@mylemma}{lemma}{Lemma}
%\Crefname{tcb@cnt@mylemma}{lemma}{Lemma}
\makeatother


\newtcbtheorem[auto counter, number within=section]{mytheo}{Theorem}%
{colback=EmoryBlue!5,colframe=EmoryBlue, fonttitle=\bfseries}{thm}

\newtcbtheorem[auto counter, number within=section]{mylemma}{Lemma}%
{colback=gray!5,colframe=gray, fonttitle=\bfseries}{lem}

\newtcbtheorem[auto counter, number within=section]{mycorollary}{Corollary}%
{colback=gray!5,colframe=gray, fonttitle=\bfseries}{cor}

\newtcbtheorem[auto counter, number within=section]{myalg}{Algorithm}%
{colback=white!5,colframe=black, fonttitle=\bfseries}{alg}

\newtcbtheorem[auto counter, number within=section]{myexam}{Example}%
{colback=white!5,colframe=EmoryGold, fonttitle=\bfseries}{exam}

\usepackage{xcolor}
\usepackage{amsthm}
\usepackage{framed}

\theoremstyle{plain}% default

\theoremstyle{definition}
\newtheorem{protoactionitems}{Action Items}[section]
\newenvironment{actionitems}
   {\colorlet{shadecolor}{lightgray!50}\begin{shaded}\begin{protoactionitems}}
   {\end{protoactionitems}\end{shaded}}


\newcommand{\powerset}[1]{\rho(#1)}


\newcommand{\eqdef}{\mathrel{\mathop:}=}
\newcommand{\starM}{\star_{\bfM}}
\newcommand{\starQ}{\star_{\bfQ^H}'}
\newcommand{\starQperp}{\star_{\bfQ_{\perp}^H}'}
%\newcommand{\starQ}{\star_{\bfQ}'}
%\newcommand{\starQperp}{\star_{\bfQ_{\perp}}'}
\newcommand{\starMP}{\star_{\bfM}^\dagger}


% additional operators
\DeclareMathOperator*{\trace}{trace}
\DeclareMathOperator*{\rank}{rank}
\DeclareMathOperator*{\diag}{diag}
\DeclareMathOperator*{\argmax}{arg\ max}
\DeclareMathOperator*{\argmin}{arg\ min}
\DeclareMathOperator{\trank}{\starM-rank}
\DeclareMathOperator*{\subjectto}{s.t.}
\DeclareMathOperator{\projtrank}{\starQ-rank}
\DeclareMathOperator{\projtrankperp}{\starQperp-rank}
\DeclareMathOperator*{\myStore}{stored}
\DeclareMathOperator*{\myValue}{value}
\DeclareMathOperator*{\myVec}{vec}
\DeclareMathOperator{\myFold}{fold}
\DeclareMathOperator{\mySqueeze}{squeeze}
\DeclareMathOperator{\myFloat}{float}
\DeclareMathOperator{\myCol}{col}
\DeclareMathOperator{\myNull}{null}
\DeclareMathOperator{\mySpan}{span}
\DeclareMathOperator{\Stiefel}{St}
\DeclareMathOperator{\GL}{GL}
\DeclareMathOperator{\myVar}{var}
\DeclareMathOperator*{\myBdiag}{bdiag}




% colors
\newcommand{\red}[1]{{\color{red} #1}}
\newcommand{\blue}[1]{{\color{blue} #1}}
\newcommand{\black}[1]{{\color{black} #1}}
\newcommand{\cyan}[1]{{\color{cyan} #1}}
\newcommand{\magenta}[1]{{\color{magenta} #1}}
\newcommand{\white}[1]{{\color{white} #1}}

% MATLAB DEFAULT COLORS
\definecolor{mycolor1}{rgb}{0.00000,0.44700,0.74100}%
\definecolor{mycolor2}{rgb}{0.85000,0.32500,0.09800}%
\definecolor{mycolor3}{rgb}{0.92900,0.69400,0.12500}%
\definecolor{mycolor4}{rgb}{0.49400,0.18400,0.55600}%
\definecolor{mycolor5}{rgb}{0.46600,0.67400,0.18800}%
\definecolor{mycolor6}{rgb}{0.30100,0.74500,0.93300}%

\newcommand{\question}[1]{\textnormal{\blue{#1}}}
\newcommand{\answer}[1]{\textnormal{\magenta{#1}}}

% math symbols for vectors and matrices
\def\mydefb#1{\expandafter\def\csname bf#1\endcsname{\mathbf{#1}}}
\def\mydefallb#1{\ifx#1\mydefallb\else\mydefb#1\expandafter\mydefallb\fi}
\mydefallb aAbBcCdDeEfFgGhHiIjJkKlLmMnNoOpPqQrRsStTuUvVwWxXyYzZ\mydefallb

% mathbb symbols for real numbers, integers, etc.
\def\mydefb#1{\expandafter\def\csname #1bb\endcsname{\mathbb{#1}}}
\def\mydefallb#1{\ifx#1\mydefallb\else\mydefb#1\expandafter\mydefallb\fi}
\mydefallb aAbBcCdDeEfFgGhHiIjJkKlLmMnNoOpPqQrRsStTuUvVwWxXyYzZ\mydefallb

% mathcal symbols for script letters 
\def\mydefb#1{\expandafter\def\csname #1cal\endcsname{\mathcal{#1}}}
\def\mydefallb#1{\ifx#1\mydefallb\else\mydefb#1\expandafter\mydefallb\fi}
\mydefallb aAbBcCdDeEfFgGhHiIjJkKlLmMnNoOpPqQrRsStTuUvVwWxXyYzZ\mydefallb

% mathbold symbols for tensors letters 
\def\mydefb#1{\expandafter\def\csname T#1\endcsname{\boldsymbol{\mathcal{#1}}}}
\def\mydefallb#1{\ifx#1\mydefallb\else\mydefb#1\expandafter\mydefallb\fi}
\mydefallb aAbBcCdDeEfFgGhHiIjJkKlLmMnNoOpPqQrRsStTuUvVwWxXyYzZ\mydefallb

% math bold font for greek letters
\def\mydefgreek#1{\expandafter\def\csname bf#1\endcsname{\text{\boldmath$\mathbf{\csname #1\endcsname}$}}}
\def\mydefallgreek#1{\ifx\mydefallgreek#1\else\mydefgreek{#1}%
   \lowercase{\mydefgreek{#1}}\expandafter\mydefallgreek\fi}
\mydefallgreek {alpha}{Alpha}{beta}{Beta}{gamma}{Gamma}{delta}{Delta}{epsilon}{Epsilon}{zeta}{Zeta}{eta}{Eta}{theta}{Theta}{iota}{Iota}{kappa}{Kappa}{lambda}{Lambda}{mu}{Mu}{nu}{Nu}{omicron}{Omicron}{pi}{Pi}{rho}{Rho}{sigma}{Sigma}{tau}{Tau}{upsilon}{Upsilon}{phi}{Phi}{xi}{Xi}{chi}{Chi}{psi}{Psi}{omega}{Omega}\mydefallgreek


% commenting commmands
\newcommand{\liznote}[1]{{\color{blue}Liz: #1}}
\newcommand{\myQuestion}[1]{{\color{black}{\bf Question:} #1}}
\newcommand{\myAnswer}[1]{{\color{black} {\bf Answer:} #1}}
 % \usepackage{subfigure}
% \captionsetup[subfigure]{labelformat=simple}
\usepackage{subcaption}
% \usepackage{enumitem}
%\newlist{todolist}{itemize}{2}
%\setlist[todolist]{label=$\square$}


\definecolor{purple1}{RGB}{135, 225, 0} 
\definecolor{purple2}{RGB}{193, 252, 129} 
\definecolor{purple3}{RGB}{228, 244, 210} 

\definecolor{teal1}{RGB}{80, 227, 194} 
\definecolor{teal2}{RGB}{167, 246, 228} 
\definecolor{teal3}{RGB}{221, 247, 239} 


\definecolor{green1}{RGB}{234, 140, 229} 
\definecolor{green2}{RGB}{235, 177, 237} 
\definecolor{green3}{RGB}{246, 222, 244} 


\newcommand{\tube}[5]{
\def\xx{#1}
\def\yy{#2}
\def\w{#3}
\def\d{#4}
\def\c{#5}

% side
\fill[fill=\c] 	(\xx,\yy+\w) -- (\xx+\d,\yy+\w+\d) --  (\xx+\w+\d,\yy+\w+\d) --  (\xx + \w,\yy+\w) -- cycle;
\draw 	(\xx,\yy+\w) -- (\xx+\d,\yy+\w+\d) --  (\xx+\w+\d,\yy+\w+\d) --  (\xx + \w,\yy+\w);
		
% top
\fill[fill=\c!50] 	(\xx+\w,\yy+\w) -- (\xx+\w+\d,\yy+\w+\d) --  (\xx+\w+\d,\yy+\d) --  (\xx + \w,\yy) -- cycle;
\draw 		(\xx+\w,\yy+\w) -- (\xx+\w+\d,\yy+\w+\d) --  (\xx+\w+\d,\yy+\d) --  (\xx + \w,\yy);

% front		
\draw[fill=\c!75] (\xx,\yy) rectangle (\xx+\w,\yy+\w);
}