\begin{tikzpicture}

% CHOOSE DIRECTORY
\def\myDir{../results_traffic}

\def\w{3}
\def\s{0.05}



\def\myname{I}
\foreach[count=\b from 1] \i in {1, 2, 3, 60, 115, 120}{
	\ifthenelse{\b < 2}{
				\node (\myname\b) {\includegraphics[width=\w cm]{\myDir/\myname/img/feature_\i.png}};
			}{
				\pgfmathsetmacro{\c}{int(\b - 1)}
				\node[right=-0.25cm of \myname\c.east, anchor=west] (\myname\b) {\includegraphics[width=\w cm]{\myDir/\myname/img/feature_\i.png}};
			}
			
			\node[above=0.0cm of \myname\b.north] {slice $\i$};
}

\node[rotate=90, above=0.0cm of \myname1.west] {$\bfQ = \bfI$};


\def\myname{W}
\foreach[count=\b from 1] \i in {1, 2, 3, 60, 115, 120}{
	\ifthenelse{\b < 2}{
				\node[below=-0.25cm of I1.south, anchor=north] (\myname\b) {\includegraphics[width=\w cm]{\myDir/\myname/img/feature_\i.png}};
			}{
				\pgfmathsetmacro{\c}{int(\b - 1)}
				\node[right=-0.25cm of \myname\c.east, anchor=west] (\myname\b) {\includegraphics[width=\w cm]{\myDir/\myname/img/feature_\i.png}};
			}
}

\node[rotate=90, above=0.0cm of \myname1.west] {$\bfQ = \bfW$};


\def\myname{C}
\foreach[count=\b from 1] \i in {1, 2, 3, 60, 115, 120}{
	\ifthenelse{\b < 2}{
				\node[below=-0.25cm of W1.south, anchor=north] (\myname\b) {\includegraphics[width=\w cm]{\myDir/\myname/img/feature_\i.png}};
			}{
				\pgfmathsetmacro{\c}{int(\b - 1)}
				\node[right=-0.25cm of \myname\c.east, anchor=west] (\myname\b) {\includegraphics[width=\w cm]{\myDir/\myname/img/feature_\i.png}};
			}
}

\node[rotate=90, above=0.0cm of \myname1.west] {$\bfQ = \bfC^\top$};

\def\myname{Z}
\foreach[count=\b from 1] \i in {1, 2, 3, 60, 115, 120}{
	\ifthenelse{\b < 2}{
				\node[below=-0.25cm of C1.south, anchor=north] (\myname\b) {\includegraphics[width=\w cm]{\myDir/\myname/img/feature_\i.png}};
			}{
				\pgfmathsetmacro{\c}{int(\b - 1)}
				\node[right=-0.25cm of \myname\c.east, anchor=west] (\myname\b) {\includegraphics[width=\w cm]{\myDir/\myname/img/feature_\i.png}};
			}
}

\node[rotate=90, above=0.0cm of \myname1.west] {$\bfQ = \bfU_3$};


\end{tikzpicture}



%\begin{tikzpicture}
%
%
%% CHOOSE DIRECTORY
%\def\myDir{results/traffic}
%
%\large
%
%
%% legend options
%\pgfplotsset{
%    compat=newest,
%    /pgfplots/legend image code/.code={%
%        \draw[mark repeat=5,mark phase=3,#1] 
%            plot coordinates {
%                (0cm,0cm) 
%                (0.3cm,0cm)
%                (0.6cm,0cm)
%                (0.9cm,0cm)
%                (1.2cm,0cm)%
%            };
%    },
%        % Global legend style
%    legend style={
%        at={(0.98, 0.02)},
%        anchor=south east,
%        legend columns=1,
%        /tikz/every even column/.append style={column sep=0.5cm},
%        font=\small
%
%    }
%}
%
%% https://copyprogramming.com/howto/filter-pgfplots-data-more-than-once-e-g-with-discard-if?utm_content=cmp-true
%
%% ad hoc approach: https://tex.stackexchange.com/questions/641588/filter-several-rows-in-addplot
%\pgfplotsset{
%    discard if/.style 2 args={
%        x filter/.append code={
%            \ifdim\thisrow{#1} pt=#2pt
%                \def\pgfmathresult{inf}
%            \fi
%        }
%    },
%    discard if not/.style 2 args={
%        x filter/.append code={
%            \ifdim\thisrow{#1} pt=#2pt
%            \else
%                \def\pgfmathresult{inf}
%            \fi
%        }
%    }
%}
%
%\def\lw{2.0}
%\def\ms{6}
%
%\pgfplotsset
%{
%m1/.style={color=mycolor1, line width=\lw pt, mark size=\ms pt, mark=o, mark options={solid, mycolor1}, mark repeat=5, only marks},
%m2/.style={color=mycolor2, line width=\lw pt, mark size=\ms pt, mark=triangle, mark options={solid, mycolor2, rotate=270}, mark repeat=1, only marks},
%m3/.style={color=mycolor4, line width=\lw pt, mark size=\ms pt, mark=square, mark options={solid, mycolor4, rotate=90}, mark repeat=1, only marks},
%m4/.style={color=mycolor5, line width=\lw pt, mark size=\ms pt, mark=diamond, mark options={solid, mycolor5}, mark repeat=1, only marks}
%}
%
%
%
%\begin{axis}[%
%scale only axis,
%width=10cm,
%height=7cm,
%at={(1.033in,0.719in)},
%scale only axis,
%axis x line=bottom, 
%axis y line=left, 
%axis line style={-}, 
% tick align=outside, % Place ticks outside the plot area
%xmin=1,
%xmax=120,
%% xtick={1e-1, 0.3162, 1e0},
%xmode=linear,
%% xticklabels={1, 50, 100, 150, 200, 250},
%scaled x ticks=false,
%xlabel={Frontal Slice Index in Transform Domain},
%ymin=-1e-2,
%ymax=1e-2,
%% ytick={1e-1, 0.3162, 1e0},
%ylabel={Color Limits},
%ymode=linear,
%xmajorgrids,
%ymajorgrids,
%    grid style={line width=1pt, draw=gray!10},
%    major grid style={line width=1pt,draw=gray!50},
%% can do things to make legend look nice
% legend cell align={left},
% legend style={at={(axis cs:1,1)}, anchor=south, legend columns=1, legend transposed}, 
%%  legend entries={$\bfZ$, $\bfC$, $\bfW$, $\bfI$}
%]
%
%
%
%
%\foreach[count=\c from 1] \M in {I, C, W, Z}{
%
%\edef\temp{
%
%\noexpand \addplot [m\c, mark repeat=5] 
%	table [x=slice, y=clim_min, col sep=comma] {\myDir/\M/feature_clim.csv};
%
%}\temp
%	
%}
%
%\foreach[count=\c from 1] \M in {I, C, W, Z}{
%
%\edef\temp{
%
%\noexpand \addplot [m\c, mark repeat=5] 
%	table [x=slice, y=clim_max, col sep=comma] {\myDir/\M/feature_clim.csv};
%
%}\temp
%	
%}
%
%\end{axis}
%
%
%\end{tikzpicture}

